\section{Problems before 4.26.2020}

\subsection{Dimension counts}

\textit{Exercise 1.3.10, Page 47}

When Dr. Tao was introducing the eigenvalue replusion phenomenon, he first try to show that it is a generic behaviour that a Hermitian matrix has simple spectrum, for which he used this \textit{Excercise 1.3.10}.

\begin{proposition}
    Suppose $n\geq 2$, the space of Hermitian matrices with at least one repeated eigenvalue has codimension-3 in the space of all Hermitian matrices.
     And the space of real symmetric matrices with at least one repeated eigenvalue has codimension 2 in the space of all symmetric matrices.
\end{proposition}


But first thing we notice that the space of all the Hermitian matrix with at least one repeated eigenvalue is not a linear space anymore.  
So it is hard for me to work with such kind of space or dimension in proving. May I just for now interprete it as the least number it takes to describe such a Hermitian matrix? (Which I tried to prove it in \ref{eigenvalue deformation}). Or just forget about why and go back to it after I know rigorous definition about it?

And when he said this space has codimension-3 in the space of all Hermitian matrices, I wonder when the size of matrix $n$ goes large, the dimension of Hermitian matrix grows in speed of $n^2$. Why can a "subspace" of codimension-3 show that this is somehow generic behaviour? 
